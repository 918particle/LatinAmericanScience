\title{Midterm - INTD262}
\author{Dr. Jordan Hanson - Whittier College Dept. of Physics and Astronomy}
\date{\today}
\documentclass[10pt]{article}
\usepackage[a4paper, total={18cm, 27cm}]{geometry}
\usepackage{outlines}
\usepackage{hyperref}
\usepackage{graphicx,subcaption}
\begin{document}
\maketitle

\section{Unit 3}

\begin{enumerate}
\item Recall the fascinating story about psychological research, in which the author shares that 67 percent of psychologists who were asked to share there data did not share it. (a) Were the rates of error higher or lower in the studies for which the authors did not share data? (b) In whose favor were the errors? \\ \vspace{0.5cm}
\item ``Recent research in behavioral economics has shown that groups are often better than individuals at finding errors in reasoning.'' (a) Why do you think this is the case? (b) Can you give an example of the wisdom of crowds thus far in our study of Latin American science? \\ \vspace{2cm}
\item Recall the story of cold fusion. (a) List three facets of the peer review process that went wrong in this episode. (b) How long, from start to finish, did it take for the scientific community to sort out the errors in the cold fusion research? \\ \vspace{2cm}
\item Note that we encountered several examples of viceregal engineers becoming Latin American leaders. (a) What are some examples of professions that involved modern technical skill in R\'{i}o de la Plata and Per\'{u}? (b) What is the primary profession of modern US leaders, for example, elected to The United States Congress? \\ \vspace{1.5cm}
\item Jos\'{e} Mariano Moci\~{n}o and others were ordered by the Mexican viceroy on an expedition to Nootka Island. What was the purpose of the expedition? (Take INTD255 to learn more!) \\ \vspace{1cm}
\item In Per\'{u}, we must take note of the work of Hip\'{o}lito Un\'{a}nue.  (a) What are some of his other scientific contributions? (b) In Nueva Granada, we must take note of the work of Jos\'{e} Celestino Mutis. What are some of his main contributions? \\ \vspace{1.5cm}
\item (a) When did Latin American wars of independence begin, approximately? (b) Give some examples of scientists and engineers who fought and died for their countries. \\ \vspace{1cm}
\end{enumerate}

\section{Unit 4}

\begin{enumerate}
\item How long after Semmelweis’s solution to childbed fever was germ theory introduced? \\ \vspace{1cm}
\item Where did the practice of autopsies begin? In what way does performing an autopsy fit with the scientific
attitude? \\ \vspace{1cm}
\item (a) Do you think the discovery of penicillin was an accident?  Why or why not? (b) Louis Pasteur is quoted as saying ``chance favors the prepared mind.'' What did he mean by this? (c) In light of (a) and (b) do you regard the discovery of cinchona as accidental or scientific? \\ \vspace{1.5cm}
\item What event catelyzed the formation of the Establecimiento de Ciencias de M\'{e}dicas in 1833? \\ \vspace{1cm}
\item (a) List some reasons the authors give to explain why medical reforms were slow to materialize in Nueva Granada, relative to the struggle for reform in Nueva Espa\~{n}a. (b) Who led the medical reform process in Nueva Granada in the 18th century? (c) When and where was the Facultad de Medicina reestablished in Nueva Granada, and what happened next? \\ \vspace{2cm}
\item (a) How many medical schools were there in Brazil in the eighteenth century? (b) What happened to the Portuguese Crown in 1807? What influence did this have on medical reform? \\ \vspace{1cm}
\item As the generation of doctors in Columbia returned from France in the late 19th century, what three cultural institutions did they establish to enhance medical practice? \\ \vspace{1cm}
\item \textbf{Triangulation} Suppose you observe a distant mountain from a flat plain.  Suppose you walk a baseline of 1 km, perpendicular to the direction towards the mountain.  The difference between the compass headings to the mountain at either end of the baseline is 5 degrees.  How far away is the mountain? \\ \vspace{1.5cm}
\item \textbf{Latitude and Longitude} (a) Suppose two cities lie along a constant line of longitude.  If we measure a change of 30 minutes (0.5 degree latitude) between them, how far apart are they, in km? (b) Suppose two cities lie along a constant latitude of 45 degrees North.  If they are 600 km apart, what is the change in longitude between them? \\ \vspace{1.5cm}
\end{enumerate}

\section{Unit 5}

\begin{enumerate}
\item (a) When were the first medical journals published in Columbia? (Give a few examples). (b) Compare this time frame to the publication of the first mining, chemistry, and physics journals in Mexico. (c) How, or through whom, were these journals connected to medical schools in Columbia? \\ \vspace{1.5cm}
\item In 1833, two Enlightenment period institutions were merged into the beginnings of a modern medical school in Mexico. What were the three institutions? \\ \vspace{1cm}
\item Consider our major in kinesiology and nutrition science (KNS). To what extent would we consider this medicine, in the absence of modern germ theory?  That is, are there other holistic forms of medical development we encountered in Latin American history besides vaccines and drugs that fight bacteria and viruses? \\ \vspace{1.75cm}
\item (a) What historical event in 1808 led to the creation of the first medical and surgical schools in Brazil (b) When did Brazil declare independence from Portugal? (c) How long after independence did the Brazilians
introduce modern reforms into the medical schools in Bahia and Rio de Janeiro? \\ \vspace{1.75cm}
\item In Columbia, the Escuela de Medicina was founded in 1865. It was centered on hospital-based anatomy and
physiology. Consider the following quote from the text: ``The second phase is notable for the slow progress
of laboratory-based medicine, especially etiopathological procedures. Its final stage, staring in the 1950s, is
defined by the introduction of Flexnerian reforms from North American technological medicine.'' What does
quote mean by Flexnerian? Think back to our reading in The Scientific Attitude. \\ \vspace{1.75cm}
\item (a) What was a major driver of modern epidemiology in 19th Century Brazil? (b)
What was the purpose of the Tropicalist School of medicine? (c) When did the bubonic plague enter Brazil,
and how did the Brazilians respond?
\end{enumerate}

\end{document}
