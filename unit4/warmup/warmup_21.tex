\title{Tuesday Warm-Up (INTD262)}
\author{Dr. Jordan Hanson - Whittier College Dept. of Physics and Astronomy}
\date{\today}
\documentclass[12pt]{article}
\usepackage[margin=1.25cm]{geometry}
\usepackage{hyperref}
\begin{document}
\maketitle
\small

\section{Chapter 6 of \textit{Science in Latin America}}

Here we resume our analysis of the history of medical modernization in Latin America from the 19th to the 20th centuries.  The modernization process sprung from Enlightenment reforms that introduced scientific medicine into the colonies, albeit unevenly.  The Enlightenment system traveled from post-revolutionary France to Latin America, from medical students who trained in Europe.

\subsection{Medical Schools and Health Policy}

\begin{enumerate}
\item (a) Who was first El Presidente of Gran Columbia?  In what way did he shift the responsibility of medical education, and to whom? (b) After 1850, colleges and universities were closed, and all professions were declared ``free,'' meaning no title was required to practice them.  Where did the next generation of physicians travel from Columbia to attain their medical training? (c) When this generation returned, what three actions did they perform to restart ``official'' medicine, apart from \textit{empirical} doctors? \\ \vspace{2cm}
\item (a) When were the first medical journals published in Columbia?  (Give a few examples).  (b) Compare this time frame to the publication of the first mining, chemistry, and physics journals in Mexico. (c) How, or through whom, were these journals connected to medical schools in Columbia? \\ \vspace{2cm}
\item In 1833, two Enlightenment period institutions were merged into the beginnings of a modern medical school in Mexico.  (a) What were the three institutions? (b) Consider our major in kinesiology and nutrition science (KNS).  To what extent would we consider this medicine, in the absence of modern germ theory? \\ \vspace{1cm}
\item (a) What historical event in 1808 led to the creation of the first medical and surgical schools in Brazil? (b) When did Brazil declare independence from Portugal? (c) How long after independence did the Brazilians introduce modern reforms into the medical schools in Bahia and Rio de Janeiro? \\ \vspace{1cm}
\end{enumerate}

\subsection{Health Research}

\begin{enumerate}
\item In Columbia, the Escuela de Medicina was founded in 1865.  It was centered on hospital-based anatomy and physiology. Consider the following quote from the text: ``The second phase is notable for the slow progress of laboratory-based medicine, especially etiopathological procedures.  Its final stage, staring in the 1950s, is defined by the introduction of Flexnerian reforms from North American technological medicine.''  What does quote mean by \textbf{Flexnerian}?  Think back to our reading in \textit{The Scientific Attitude.} \\ \vspace{1.5cm}
\item Why did the medical personnel involved in the \textit{Gaceta} and La Escuela Medicina move en masse to the Universidad Nacional in Columbia? \\ \vspace{1.5cm}
\item Discuss the tension between physiopathology and etiopathology in Columbian medicine in the late 19th century.  \\ \vspace{1cm}
\item According to the authors, (a) what was a major driver of modern epidemiology in 19th Century Brazil? (b) What was the purpose of the Tropicalist School of medicine? (c) When did the bubonic plague enter Brazil, and how did the Brazilians respond? \\ \vspace{1.5cm}
\end{enumerate}

\end{document}
