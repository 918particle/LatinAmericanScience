\title{Thursday Warm-Up (INTD262)}
\author{Dr. Jordan Hanson - Whittier College Dept. of Physics and Astronomy}
\date{\today}
\documentclass[12pt]{article}
\usepackage[margin=1.25cm]{geometry}
\usepackage{hyperref}
\begin{document}
\maketitle
\small

\section{Chapter 6 of \textit{Science in Latin America}}

\begin{enumerate}
\item Which of the following Mexican institutions initially resisted Enlightenment period medical reforms?
\begin{itemize}
\item A: Real Jard\'{i}n Bot\'{a}nico
\item B: Real Tribunal del Protomedicato
\item C: Real Escuela de Cirug\'{i}a de M\'{e}xico
\item D: Hospital Real de San Jos\'{e} de Lisboa
\end{itemize}
\item What event catelyzed the formation of the Establecimiento de Ciencias de M\'{e}dicas in 1833? \\ \vspace{0.5cm}
\item (a) List some reasons the authors give to explain why medical reforms were slow to materialize in Nueva Granada, relative to the struggle for reform in Nueva Espa\~{n}a. (b) Who led the medical reform process in Nueva Granada in the 18th century? (c) When and where was the Facultad de Medicina reestablished in Nueva Granada, and what happened next? \\ \vspace{1.5cm}
\item (a) How many medical schools were there in Brazil in the eighteenth century? (b) What happened to the Portuguese Crown in 1807? What influence did this have on medical reform? \\ \vspace{1cm}
\item In 1850, a law was passed in Gran Columbia (formerly Nueva Granada) that dissolved all the universities.  Given what we know of the scientific attitude, what effect does this have on a field like medicine? \\ \vspace{1cm}
\item As the generation of doctors in Columbia returned from France in the late 19th century, what three cultural institutions did they establish to enhance medical practice?
\end{enumerate}

\end{document}
