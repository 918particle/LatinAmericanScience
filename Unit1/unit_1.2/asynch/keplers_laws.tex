\title{INTD290: Number Systems in pre-Columbian Context}
\author{Dr. Jordan Hanson - Whittier College Dept. of Physics and Astronomy}
\date{\today}
\documentclass[10pt]{article}
\usepackage[a4paper, total={18cm, 27cm}]{geometry}
\usepackage{outlines}
\usepackage{hyperref}
\usepackage{graphicx}
\begin{document}
\maketitle

\section{How to Submit this Assignment}

Once you answer the questions, take a picture of your work and convert it to a PDF.  Submit the PDF to the assignment link on Moodle.

\section{Introduction}

For this asynchronous assignment, we will be using something called a \textit{Physics Educational Technology} simulation, or PhET simulation.  For an introduction to this tool, please follow this link to a tutorial video by one of my colleagues: \\

\url{https://youtu.be/m6e2y4fef1I}

\section{The Simulation}

To find this simulation, which teaches us how gravity and planetary orbits work, follow this link: \\

\url{https://phet.colorado.edu/en/simulation/gravity-and-orbits}

\section{The Basics: circular and elliptical orbits}

\textit{Instructions:}
\begin{enumerate}
\item Starting with the link above, press the ``to scale'' option at the bottom of the screen.  Chose the option with the star and planet.
\item Activate the path and grid options at right.
\item Click the play button and allow the planet to rotate through 360 degrees, all the way around the star.  You can speed up or slow down the motion, which is just governed by gravity, with the controls.
\item Use the yellow measuring tape tool at right to measure two distances: (a) the distance from the star to the path of the planet on the \textit{right}, and (b) the distance from the star to the path of the planet on the \textit{left.}  Are they the same number?
\item What would be true of the numbers if the orbit was perfectly circular? 
\end{enumerate}
\vspace{1cm}

\section{Gravity}

\textit{Instructions:}
\begin{enumerate}
\item Using the controls at right, display the direction of the force of gravity.
\item What happens to the path of the planet if you deactivate gravity?
\item What happens to the force of gravity if you leave it activated, but click and drag the planet farther from the star?
\item Display the velocity with the control at right.  Reveal what happens if you let the planet follow one orbit, and then pause, and then change the length of the velocity arrow.  This corresponds to changing the speed of the planet.  (Changing the direction of the arrow changes the direction of the velocity).
\end{enumerate}
\vspace{1.5cm}

\section{Kepler's Laws}

\textit{Instructions:}
\begin{enumerate}
\item Now that you can see how to control the system using velocity, force, and distance from star, try to make an orbit that is nearly circular.  Show that the radius of the orbit is almost the same when measured at different places (it should be the same number all the way around for a circle, but this might be challenging).
\item For your circular orbit, determine what happens if you change the mass of the planet (controls at right).  Answer this question: does the rate at which something accelerates downward due to gravity here on Earth depend on its mass?  Is it different for planets?
\item Finally, tweak your orbit so that it is elliptical.  Using the ruler and grid, find the area of a triangle swept out by the orbit when it is going faster (nearer to the star).  The planet needs some number of days to sweep out this area.  Find a different triangle on the other side of the orbit that requires the same number of days.  Can you show that these triangles have the same area?  \textit{This is Kepler's 2nd Law.}
\end{enumerate}

\end{document}
