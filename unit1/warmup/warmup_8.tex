\title{Warm-Up, September 19th, 2024 (INTD262)}
\author{Dr. Jordan Hanson - Whittier College Dept. of Physics and Astronomy}
\date{\today}
\documentclass[12pt]{article}
\usepackage[margin=1.25cm]{geometry}
\usepackage{hyperref}
\usepackage{graphicx}
\usepackage{amsmath}
\usepackage{subcaption}
\begin{document}
\maketitle

\section{Chapter 3 of \textit{The Scientific Attitude}}

\begin{enumerate}
\item What are the two tenets of the scientific attitude, or ethos, according to the author? \\ \vspace{0.5cm}
\item As far as quotes are concerned, this one is deep.
\begin{quotation}
This is not to say that science is solely justified (or not) by what it does.  Practice is important but it is not the only thing that matters in judging science.  I say this because one strain of argument against the methodological approach to the philosophy of science over the last few decades has been that, because science does not always follow the precepts that the logic of science would dictate, science must be no better or worse than any other form of inquiry.  I believe this conclusion to be misguided ... I do not believe that science is hopelessly subjective either.  Although objectivity is important, values play a role by guiding our practice and keeping us on track.  This is to say that even though the practice of science may sometimes fall short, it is still possible to justify science as a whole based on the goals of its aspiration.
\end{quotation}
\begin{itemize}
\item At other points in Ch. 3, the author points to the \textit{creed} of science, the proper values of science (including being humble, earnest, open-minded, honest, curious, and self-critical), and that the actions of scientists are judged not by the individual, but the scientific \textit{community.}  Does this sort of social structure remind you of other social structures?  Explain in detail. \\ \vspace{1.5cm}
\item Recall the story of Ignaz Semmelweis and antiseptic handwashing in maternity wards.  Discuss how the scientific attitude was applied in this situation. \\ \vspace{1cm}
\item Recall the story of the false discovery of cold fusion.  (a) Discuss how the scientific attitude was not applied in this situation. (b) Now select a piece of science from Latin American history that we have encountered thus far, and apply the criteria of the scientific attitude to it.  \\ \vspace{1cm}
\end{itemize}
\end{enumerate}

\end{document}
