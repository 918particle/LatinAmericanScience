\title{Thursday Warm-Up (INTD262)}
\author{Dr. Jordan Hanson - Whittier College Dept. of Physics and Astronomy}
\date{\today}
\documentclass[12pt]{article}
\usepackage[margin=1.25cm]{geometry}
\usepackage{hyperref}
\usepackage{graphicx}
\usepackage{amsmath}
\usepackage{subcaption}
\begin{document}
\maketitle

\small

\section{Chapter 3 of \textit{Science in Latin America}}

\subsection{Copernican and Newtonian Teaching in the Viceroyalty of New Granada}

\begin{enumerate}
\item While ''Copernican and Newtonian'' theories were yet to be adopted in many universities in Nueva Granada in the mid-1700s, these ideas did take route in institutions that needed them.  (a) What were some of these institutions? (b) Can you draw a scientific parallel to any institutions we have today ? \\ \vspace{0.5cm}
\item Scientific expeditions throughout Nueva Granada began in the mid-1700s.  In 1741, Jos\'{e} Cassani published work on the flora of the Orinoco, and Jos\'{e} Gumilla published \textit{Orinoco ilustrado}.  In the 1750s, Filipo Salvatore Gilij charted the Meta and Orinoco rivers using modern scientific techniques. (a) What did all these scientists have in common? (b) How did this enable them to disseminate their findings? \\ \vspace{0.5cm}
\item In Chapter 3 of \textit{Science in Latin America}, we encounter the following quote:
\begin{quotation}
\textit{La Universidad Gegoriana} in Quito alone had ``seventy-one foreign professors teaching at the university ... Native professors were twenty-one, of whom five were from Loja, four from Quito, three from Guayas, three from Cuenca, three from Riobamba, two from Ibarra, and one from Ambato.'' ... As a consequence, it is not strange that in a center of cultural ferment such as Quito, intellectual Jesuits were most closely linked to the Franco-Spanish geodetic mission directed by La Condamine and Jorge Juan.
\end{quotation}
(a) What scientific transition began to take place as a result of the interaction between foreign and Ecuadorian professors? (b) What can we infer about the ratio of the native professors at the university? (c) Consider Father Fransisco Javier Aguilar, who taught physics and mathematics at Universidad Gregoriana.  He taught no less than five world systems, and focused on three: Ptolemaic, Copernican, and Tychonic.  What distinguished these? \\ \vspace{0.5cm}
\end{enumerate}

\subsection{Jos\'{e} Celestino Mutis and Juan de Hospital Declare Themselves Copernicans}

\begin{enumerate}
\item In 1767, Mutis published \textit{Reflexiones sobre el sistema tyc\'{o}nico}. (a) What were the main points of this publication?  (b) Was it considered controversial?
\end{enumerate}

\end{document}
