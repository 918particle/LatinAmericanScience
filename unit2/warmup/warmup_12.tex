\title{Tuesday Warm-Up (INTD262)}
\author{Dr. Jordan Hanson - Whittier College Dept. of Physics and Astronomy}
\date{\today}
\documentclass[12pt]{article}
\usepackage[margin=1.25cm]{geometry}
\usepackage{hyperref}
\usepackage{graphicx}
\usepackage{amsmath}
\usepackage{subcaption}
\begin{document}
\maketitle

\small

\section{Article on Venus Transit - X. L\'{o}pez Medell\'{i}n, C. G. Rom\'{a}n-Z\'{u}\~{n}iga, I. Engstrand, M. Alvarez P\'{e}rez, M. Moreno Corral}

\begin{enumerate}
\item (a) Why did Venus cross the Sun in 1631, 1639, but then not again until 121.5 years later?  (b) Why did Chappe d'Auteroche select the Northern part of Nueva Espa\~{n}a for astronomical observations in 1769? \\ \vspace{0.5cm} 
\item When Joaqu\'{i}n Vel\'{a}zquez de Le\'{o}n and Jos\'{e} de G\'{a}lvez arrived in Baja California, they remained there for three years.  (a) What types of measurements did they make? (b) How did this improve local knowledge of Nueva Espa\~{n}a? (c) Vel\'{a}zquez de Le\'{o}n communicated with Chappe d'Auteroche that he would help with the Venus transit measurements, and d'Auteroche suggested that Vel\'{a}zquez de Le\'{o}n remain in Real de Santa Ana, while d'Auteroche would work in San Jos\'{e} del Cabo.  What happened as a result? \\ \vspace{1cm}
\item In 2012, expedition members attempted to find the observation point of Vel\'{a}zquez de Le\'{o}n, to repeat the measurements.  According to documents, it was near the formal warehouse of Real de Santa Ana and the surrounding mining camps.  Who explained to the expedition members that the main building at El Rancho D\'{a}til, an old adobe house, was in fact the ancient warehouse? \\ \vspace{1cm}
\item The authors write that they believe they stood where Vel\'{a}zquez de Le\'{o}n made the original measurement, and observed Venus cross the Sun within a minute of the predicted start time.  However, their location was a few hundred meters away from the predicted location in manuscripts from the 1700s.  Why? \\ \vspace{1cm}
\end{enumerate}

\section{M. P. Ramos-Lara: Contribuciones de astr\'{o}nomos mexicanos al estudio de auroras boreales de baja latitud entre 1789 y 1791}

\begin{enumerate}
\item What is one interesting fact that you derived from the paper on Mexican astrophysics of the aurora borealis?
\end{enumerate}

\end{document}
