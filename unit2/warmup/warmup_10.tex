\title{Warm-Up, October 1st, 2024 (INTD262)}
\author{Dr. Jordan Hanson - Whittier College Dept. of Physics and Astronomy}
\date{\today}
\documentclass[12pt]{article}
\usepackage[margin=1.25cm]{geometry}
\usepackage{hyperref}
\usepackage{graphicx}
\usepackage{amsmath}
\usepackage{subcaption}
\begin{document}
\maketitle

\small

\section{Chapter 4 of \textit{The Scientific Attitude}}

\begin{enumerate}
\item Restate \textit{modus tollens}, the key finding of Karl Popper, in your own words. \\ \vspace{0.5cm}
\item Restate \textit{the scientific attitude}, the key finding of Lee McIntyre, in your own words. \\ \vspace{0.5cm}
\item Recall the work of Larry Laudan.  (a) What were some of the main claims made by Prof. Laudan with regard to the demarcation problem? (b) Do you think there is such a thing as the scientific method?  What would \textit{your} scientific method be? \\ \vspace{0.5cm}
\item (a) Is \textit{modus tollens} (the idea of Karl Popper) both \textit{necessary} and \textit{sufficient} to demarcate science from other scholarship? (b) Give a counter-example from the physics of the solar system. (b) Give a counter example from medicine. \\ \vspace{0.5cm}
\item Consider the following scientific claim: anyone born between 1945 and 1991 has Strontium-90 in their bones.  (a) In logical terms, is a positive Strontium-90 observation in the bones sufficient as proof that someone is born between those dates? (b) In logical terms, is a positive Strontium-90 observation in the bones necessary as proof that someone is born between those dates? (c) Is the following true: Strontium-90 is observed in the bones if and only if one was born between 1945-1991? \\ \vspace{1cm}
\item Evaluate the logical truth of this claim: ``anti-vaccination campaigns do not have the scientific attitude, therefore these are not scientific endeavors.'' \\ \vspace{1cm}
\item \textbf{Application to Latin Amercian scientific history.} Discuss one example we have encountered from our scientific history that should count as science, even though it has not traditionally been considered scientific.
\end{enumerate}

\end{document}
