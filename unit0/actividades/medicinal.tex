\title{INTD262: Herbal Medicine in Mesoamerica}
\author{Dr. Jordan Hanson - Whittier College Dept. of Physics and Astronomy}
\date{\today}
\documentclass[12pt]{article}
\usepackage[margin=1.5cm]{geometry}
\usepackage{hyperref}
\usepackage{graphicx}
\usepackage{amsmath}
\begin{document}
\maketitle

\section{Introduction}

In this activity, we will match herbal remedies common to Mesoamerican peoples and European colonials to diseases and symptoms.  Our matching will be based on \textit{Science in Latin America}, and other sources.

\section{Common Diseases and Medical Issues, Herbal Treatments}
\label{sec:1}

Consider the following list of common diseases and medical situations:

\begin{enumerate}
\item \textbf{Diarrhea}: having at least three loose, liquid, or watery bowel movements in a day. It often lasts for a few days and can result in dehydration due to fluid loss.
\item \textbf{Malaria}: a mosquito-borne infectious disease that affects vertebrates. Human malaria causes symptoms that typically include fever, fatigue, vomiting, and headaches. In severe cases, it can cause jaundice, seizures, coma, or death.
\item \textbf{Bone Fracture}: partial or complete break in the continuity of any bone in the body. In more severe cases, the bone may be broken into several fragments, known as a comminuted fracture. A bone fracture may be the result of high force impact or stress, or a minimal trauma injury as a result of certain medical conditions that weaken the bones, such as osteoporosis, where the fracture is termed \textit{pathologic}.
\item \textbf{Dysentery}: a type of gastroenteritis that results in bloody diarrhea.  Other symptoms may include fever, abdominal pain, and a feeling of incomplete defecation.  Complications may include dehydration.
\end{enumerate}

Consider the following list of common herbal treatments:
\begin{enumerate}
\item \textbf{Tzipipatli} root
\item \textbf{Mixture of atole} (\textit{atolli}) with chia.  With chocolate (\textit{chocolatl}): \textit{champurrado}.
\item \textbf{Xalxocotl}, boiled leaves.  Leaves of the guava tree.
\item \textbf{Zacacili} root
\item \textbf{Itzaczazalic}
\item \textbf{Tememetatl} root
\item \textbf{Cinchona} bark
\end{enumerate}

\section{Match Treatments to Medical Issues}

Complete the exercise in Tab. \ref{tab:1} below.

\begin{table}[hb]
\centering
\begin{tabular}{| c | c | c |}
\hline
\textbf{Medical Issue} & \textbf{Treatment} & \textbf{Healing} \\ \hline
Diarrhea & \hspace{7cm} & \hspace{7cm} \\ \hline
Malaria & \hspace{7cm} & \hspace{7cm} \\ \hline
Bone fracture & \hspace{7cm} & \hspace{7cm} \\ \hline
Dysentery & \hspace{7cm} & \hspace{7cm} \\ \hline
\hline
\end{tabular}
\caption{\label{tab:1} The four medical issues mentioned in Sec. \ref{sec:1} are listed in the left column.  Write in the middle column the treatment prescribed by Nahuatl-speaking doctors.  Using \textit{Science in Latin America,} and The Online Nahuatl Speaking Dictionary (\url{https://nahuatl.wired-humanities.org/}, try to assign a healing mechanism for each treatment.  That is, answer \textit{why} a particular treatment alleviates the medical issue.}
\end{table}

\section{Special Topic: Quinine and Malaria}

Quinine was first isolated in 1820 from the bark of a cinchona tree, which is native to Per\'{u}, and its molecular formula was determined by Adolph Strecker in 1854. The class of chemical compounds to which it belongs is thus called the cinchona alkaloids. Bark extracts had been used to treat malaria since at least 1632 and it was introduced to Spain as early as 1636 by Jesuit missionaries returning from the New World. Treatment of malaria with quinine marks the first known use of a chemical compound to treat an infectious disease.

Posted on Moodle is a peer-reviewed journal article on the historical uses of quinine and modern alternatives.  Read the article, and answer the following questions together in your group.

\begin{enumerate}
\item Is quinine currently the front-line treatment for malaria patients?  Why or why not?  Take into consideration the absorption of the compound in young patients, old patients, and pregnant mothers.  \\ \vspace{3cm}
\item Define (in your own words) the \textit{therapeutic index} (TI) for a chemical compound used to treat disease.  If you encounter this term in the paper, but do not understand its context, use ChatGPT or Google to produce definitions and graphs of TI. \\ \vspace{2cm}
\item One side effect of quinine dosage is hypoglycaemia.  What is hypoglycaemia?\footnote{Though hypoglycaemia is concerning for pregnant mothers, women who are pregnant are often unable to take medications that cross the placental barrier.  These medications can interfere with child development.  Intriguingly, quinine is also shown to cross the placental barrier, but it is still recommended to treat pregnant malaria patients in the first trimester.} \\ \vspace{1cm}
\item To where in the Spanish Colonial world was cinchona production moved after being primarily located in Latin America?  Where in today's world is quinine still a \textit{monotherapy}?  Why do you think this is the case? \\ \vspace{2cm}
\item For uncomplicated malaria, give some examples of other antibiotics that are combined with quinine to boost cure rates.  Why is basic quinine no longer as effective against forms of malaria in some parts of the world?  \\ \vspace{2cm}
\item Consider the following passage from the review article:
\begin{quotation}
Even with seven-day treatment durations, evaluations
of different quinine dosage regimens have revealed
interesting trends. Doses of 10 mg/kg/day given twice
daily for 7 days were associated with day 28 treatment
failure rates as high as 30\% [37]. Increasing the quinine
dosage to 15 mg/kg/day or 20 mg/kg/day improved
treatment outcomes, with failure rates ranging from 8\%
to 14\% [37], although potential increases in toxicity with
higher dosages are a concern.  The treatment regimen
currently recommended in sub-Saharan Africa is 10 mg/
kg of the base given 8 hourly for 7 days. This regimen
was associated with a lower rate of recurrent infections
on day 28 (6.3\%) compared to the 10 mg/kg twice daily
regimen (16.1\%) [44].
\end{quotation}
Suppose you have a young malaria patient from Latin America who weighs 25 kg.  (a) How many mg of quinine should this patient receive, and how often? (b) If you treated 100 such patients (similar region and symptoms), how many treatment failures should you expect?  (c) Repeat the calculations for sub-Saharan Africa. \\ \vspace{2cm}
\item (a) List some risks associated with malaria and pregnancy. (b) Why are quinine and clindamycin used instead of artemisinin compounds in the first trimester of pregnancy? (c) What is the primary drawback of the quinine/clindamycin combination in certain parts of the world? (d) Is there a biological arguement for quinine monotherapy in sub-Saharan Africa for pregnant women? \\ \vspace{2cm}
\item When treating severe malaria cases in children in sub-Saharan Africa, what were the results of the AQUAMAT study? What can be said about the relative availability of artemisinin type medications? \\ \vspace{1.5cm}
\item On the whole, have the bacteria that lead to malaria built substantial resistance to quinine?  Do we find examples of evidence for altered bacterial sensitivity in the article? \\ \vspace{1.5cm}
\item \textbf{Personal reflection:} What parts of this study remind you of the pandemic due to COVID-19 and subsequent efforts to treat it?
\end{enumerate}

\end{document}
