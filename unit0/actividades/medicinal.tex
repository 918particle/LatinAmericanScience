\title{INTD262: Herbal Medicine in Mesoamerica}
\author{Dr. Jordan Hanson - Whittier College Dept. of Physics and Astronomy}
\date{\today}
\documentclass[12pt]{article}
\usepackage[margin=1.5cm]{geometry}
\usepackage{hyperref}
\usepackage{graphicx}
\usepackage{amsmath}
\begin{document}
\maketitle

\section{Introduction}

In this activity, we will match herbal remedies common to Mesoamerican peoples and European colonials to diseases and symptoms.  Our matching will be based on \textit{Science in Latin America}, and other sources.

\section{Common Diseases and Medical Issues, Herbal Treatments}
\label{sec:1}

Consider the following list of common diseases and medical situations:

\begin{enumerate}
\item \textbf{Diarrhea}: having at least three loose, liquid, or watery bowel movements in a day. It often lasts for a few days and can result in dehydration due to fluid loss.
\item \textbf{Malaria}: a mosquito-borne infectious disease that affects vertebrates. Human malaria causes symptoms that typically include fever, fatigue, vomiting, and headaches. In severe cases, it can cause jaundice, seizures, coma, or death.
\item \textbf{Bone Fracture}: partial or complete break in the continuity of any bone in the body. In more severe cases, the bone may be broken into several fragments, known as a comminuted fracture. A bone fracture may be the result of high force impact or stress, or a minimal trauma injury as a result of certain medical conditions that weaken the bones, such as osteoporosis, where the fracture is termed \textit{pathologic}.
\item \textbf{Dysentery}: a type of gastroenteritis that results in bloody diarrhea.  Other symptoms may include fever, abdominal pain, and a feeling of incomplete defecation.  Complications may include dehydration.
\end{enumerate}

Consider the following list of common herbal treatments:
\begin{enumerate}
\item \textbf{Tzipipatli} root
\item \textbf{Mixture of atole} (\textit{atolli}) with chia.  With chocolate (\textit{chocolatl}): \textit{champurrado}.
\item \textbf{Xalxocotl}, boiled leaves.  Leaves of the guava tree.
\item \textbf{Zacacili} root
\item \textbf{Itzaczazalic}
\item \textbf{Tememetatl} root
\item \textbf{Cinchona} bark
\end{enumerate}

\section{Match Treatments to Medical Issues}

Complete the exercise in Tab. \ref{tab:1} below.

\begin{table}[hb]
\centering
\begin{tabular}{| c | c | c |}
\hline
\textbf{Medical Issue} & \textbf{Treatment} & \textbf{Healing} \\ \hline
Diarrhea & \hspace{7cm} & \hspace{7cm} \\ \hline
Malaria & \hspace{7cm} & \hspace{7cm} \\ \hline
Bone fracture & \hspace{7cm} & \hspace{7cm} \\ \hline
Dysentery & \hspace{7cm} & \hspace{7cm} \\ \hline
\hline
\end{tabular}
\caption{\label{tab:1} The four medical issues mentioned in Sec. \ref{sec:1} are listed in the left column.  Write in the middle column the treatment prescribed by Nahuatl-speaking doctors.  Using \textit{Science in Latin America,} and The Online Nahuatl Speaking Dictionary (\url{https://nahuatl.wired-humanities.org/}, try to assign a healing mechanism for each treatment.  That is, answer \textit{why} a particular treatment alleviates the medical issue.}
\end{table}

\section{Special Topic: Quinine and Malaria}

Quinine was first isolated in 1820 from the bark of a cinchona tree, which is native to Per\'{u}, and its molecular formula was determined by Adolph Strecker in 1854. The class of chemical compounds to which it belongs is thus called the cinchona alkaloids. Bark extracts had been used to treat malaria since at least 1632 and it was introduced to Spain as early as 1636 by Jesuit missionaries returning from the New World. Treatment of malaria with quinine marks the first known use of a chemical compound to treat an infectious disease.

Posted on Moodle is a peer-reviewed journal article on the historical uses of quinine and modern alternatives.  Read the article, and answer the following questions:

\begin{enumerate}
\item things things
\end{enumerate}

\end{document}
