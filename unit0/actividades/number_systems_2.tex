\title{INTD262: Problem-Solving in pre-Columbian Mayan Culture}
\author{Dr. Jordan Hanson - Whittier College Dept. of Physics and Astronomy}
\date{\today}
\documentclass[12pt]{article}
\usepackage[margin=1.5cm]{geometry}
\usepackage{hyperref}
\usepackage{graphicx}
\usepackage{amsmath}
\begin{document}
\maketitle

\section{Introduction}

In this activity, we will utilize the Mayan number system to solve examples of construction and agricultural problems.  The utility of the vigesimal system will be illustrated.

\section{Review of Base-20 or Vigesimal}

Recall that in a base-20 system, the weights are powers of 20: 1 ($20^0$), 20 ($20^1$), and 400 ($20^2$). Thus, the number 4205 in vigesimal translates to

\begin{equation}
AA5 = 10 \times 20^2 + 10 \times 20^1 + 5\times 20^0 = 4000 + 200 + 5 = 4205
\end{equation}

On the right hand side of the equation, we are representing the number in decimal.  We are using the symbol ``A'' to represent the number ten, as we do in hexadecimal.

Recall Fig. \ref{fig:maya} from our prior activity.  The Mayan system has 20 digits, comprised of three symbols.  A dot represents one unit, a bar represents five, and the empty shell represents zero.  The digits are constructed vertically, with the least/most significant digits at the bottom/top, respectively.

\begin{figure}[hb]
\centering
\includegraphics[width=6cm]{figures/maya_digits.png}
\caption{\label{fig:maya} The 20 digits of the Mayan system.  The digit for 0 resembles an empty shell.  The dots are worth 1 and the bars are worth 5.  In a number larger than 19, the symbols stack vertically.}
\end{figure}

\clearpage

\section{Construction Problems}

\begin{enumerate}
\item Suppose we need to cut a log to the proper length for the beam of a house.  (a) The log is 15 meters long (think of a meter as the distance from fingertips to sternum).  Write the length in Mayan script.  (b) The length we need is 4 meters.  Write the desired length in Mayan script.  (c) Subtract 4 from 15 (in Mayan) to obtain the length we need to cut. \\ \vspace{3.5cm}
\item (a) Suppose we are looking for a tree tall enough to supply four, four-meter beams.  Write the desired height of the tree trunk (in meters, in Mayan). (b) In Mayan, four meters the correct number of times to show that the trunk is tall enough. \\ \vspace{3.5cm}
\item Suppose we are planning a pyramid that will serve a variety of cultural functions.  We know that the pyramid faces need to be four equilateral triangles. (a) Draw a diagram and determine the distance to the center of the base from a corner of the base.  If the length of the base of one side of the pyramid is $x$, work out a simple expression for the pyramid height, in terms of $x$.  (b) Treat $1/\sqrt{2}$ as 0.7, and let $x = 100$ meters.  Write the base length (in meters) in Mayan script.  Do the same for the height. (c) Suppose we have stones that are 1 meter x 1 meter x 0.7 meters.  How many stones are required for the base of the pyramid?  Show your calculations in Mayan script. \\ \vspace{7cm}
\end{enumerate}

\section{Agricultural Problems}

\begin{enumerate}
\item Suppose we are planning \textit{una milpa}, an agricultural plot designated for maize, avocados, squash, chilis, tomatoes, and other produce.  We know we need to clear a patch of land with an area of 100 meters x 100 meters.  (a) Work out the area in meters-squared, in modern notation. (b) Convert this number to Mayan script.  (c) In this particular area, with the stone and wood tools available, one adult can clear 10 meters-squared in 1 hour.  If 10 adults are working together, how many hours are needed to clear the \textit{milpa}?  Show this calculation in Mayan script. \\ \vspace{4cm}
\item Suppose we are planting crops to feed a family of six people, two parents and four children.  Suppose we need 25 percent of our 2000 daily calories from corn, and that one corn plant provides about 1,000 calories. (a) How many corn plants are necessary for our family diet for 1 day?  Work this out in Mayan script by tallying the needs of each person.  (b) Now tally the number of corn plants to be planted for 10 days. (c) Suppose two children leave the home to form their own families.  Account for the change to the \textit{milpa} by subtracting plants from the original total plants.  (d) How many plants are necessary for the family diet for 100 days?
\end{enumerate}

\end{document}
