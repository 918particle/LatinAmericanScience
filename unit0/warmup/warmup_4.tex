\title{Warm-Up, September 3rd, 2024 (INTD262)}
\author{Dr. Jordan Hanson - Whittier College Dept. of Physics and Astronomy}
\date{\today}
\documentclass[12pt]{article}
\usepackage[margin=1.5cm]{geometry}
\usepackage{hyperref}
\usepackage{graphicx}
\usepackage{amsmath}
\begin{document}
\maketitle

\section{Chapter 1 of \textit{The Scientific Attitude}}

\begin{enumerate}
\item Chapter 1 of \textit{The Scientific Attitude}:
\begin{quotation}
In 1981, the state of Arkansas passed Act 590, which required that public school teachers give ``balanced treatment'' to ``creation science'' and ``evolution science'' in the biology classroom ... [T]he curriculum was expected to concentrate only on the ``scientific evidence'' for creation science.  But was there any?  And, how precisely was creation science different from creationism?
\end{quotation}
Explain the arguments used in court to thwart Act 590 the following year. \\ \vspace{0.75cm}
\item Define the term \textit{logical positivism} and give an example of how is is used in science. \\ \vspace{0.5cm}
\item Define the term \textit{modus tollens} and give an example of how is is used in science. \\ \vspace{0.5cm}
\item Explain how \textit{modus tollens} is used in the following logic: ``If someone was born between 1945 and 1991, then they have Strontium-90 in their bones.  Gabriel does not have Strontium-90 in his bones.  Therefore, Gabriel was not born between 1945 and 1991.''  \\ \vspace{0.5cm}
\item Thomas Kuhn wrote a famous book entitled \textit{The Structure of Scientific Revolutions} (1962).  Rather than describing science as a global accumulation of progress, he argues that, sociologically, scientists move between periods of ``puzzle-solving'' within an accepted framework and revolution triggered by unavoidable experimental anomalies. (a) Give one example of a scientific revolution, and note the anomaly. (b) Do you think that the colonization of Nueva Espa\~{n}a triggered a scientific revolution?
\end{enumerate}

\end{document}
