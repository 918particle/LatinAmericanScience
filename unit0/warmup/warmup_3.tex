\title{Warm-Up, September 3rd, 2024 (INTD262)}
\author{Dr. Jordan Hanson - Whittier College Dept. of Physics and Astronomy}
\date{\today}
\documentclass[12pt]{article}
\usepackage[margin=1.5cm]{geometry}
\usepackage{hyperref}
\usepackage{graphicx}
\usepackage{amsmath}
\begin{document}
\maketitle

\section{Chapter 1 of \textit{Science in Latin America: A History}}
\small
\begin{enumerate}
\item \textbf{First Period}
\begin{enumerate}
\item Which of the following where media through which inhabitants of the Mexica empire recorded scientific observations about the natural world?
\begin{itemize}
\item A: \textit{Axolotl} (codices) and \textit{huitzitzilin} (paintings, stelae)
\item B: \textit{Amoxtl} (codices) and \textit{tlacuiloll} (paintings, stelae)
\item C: \textit{Tomatl} (plume, writing tool) and \textit{altepetl} (city-state)
\item D: \textit{Quetzal} (plume, writing tool) and \textit{huitzitzilin} (city-state)
\end{itemize}
\item Using information from \textit{Historia natural y moral de las Indias} (de Acosta), \textit{Historia general y natural de las Indias} (Oviedo), \textit{D\'{e}cadas del Nuevo Mundo} (Angler\'{i}a), \textit{Historia de Nueva Espa\'{n}a} (Hern\'{a}ndez), match the European story to the indigenous story or piece of knowledge.
\begin{itemize}
\item A: Ponce de Le\'{o}n and the Fountain of Youth
\item B: Griffins so large they capture people and calves as prey, with feathers as large as an arm.
\item C: ``A fountain running with hot water and as the water runs it turns to stone.''
\item D: ``fish that as they leave the water turn into butterflies.''
\item E: ``...a monstrous animal, with the face of a fox, a tail of a cercopithecus, ears of a bat, human hands, and feet of a monkey.'' Carries young on the belly.
\end{itemize}
\hrulefill
\begin{itemize}
\item A: A flying fish
\item B: A condor
\item C: A mercury mine
\item D: The belief about a certain river among the Lucayo and Carib indigenous
\item E: The Mexican opposum
\end{itemize}
\end{enumerate}
\item \textbf{Second Period}
\begin{enumerate}
\item Father Bernardino de Sahag\'{u}n translates from Nahuatl a description of a ``tiger'' that the indigenous say can do the following: (a) see small things even though there is fog or darkness (b) creates sounds 	``through the air'' to intimidate hunters.  What does this writing tell us about the Nahua understanding of physics? \\ \vspace{0.5cm}
\item Why did both the Spaniards and Nahuatl-speaking peoples believe that hummingbirds were connected to immortality? \\ \vspace{1cm}
\end{enumerate}
\end{enumerate}

\section{Introduction and Chapter 1 of \textit{The Scientific Attitude}}

\begin{enumerate}
\item Suppose the following statement is given: ``If someone was born between 1945 and 1991, then they have Strontium-90 in their bones.''  Which of the following statements is \textit{deductively valid}?
\begin{itemize}
\item Adam was born in 1963.  Therefore, Adam has Strontium-90 in his bones
\item Eve has Strontium-90 in her bones.  Therefore, Eve was born between 1945 and 1991.
\end{itemize}
\item What is the name of the following type of argument? ``If A, then B.  And not B.  Therefore, not A.'' \\ \vspace{0.25cm}
\item Explain how the following argument is deductively valid and follows a scientific attitude: ``If someone was born between 1945 and 1991, then they have Strontium-90 in their bones.  Gabriel does not have Strontium-90 in his bones.  Therefore, Gabriel was not born between 1945 and 1991.''  \\ \vspace{1cm}
\item Consider the following passage from Chapter 1 of \textit{The Scientific Attitude}:
\begin{quotation}
In 1981, the state of Arkansas passed Act 590, which required that public school teachers give ``balanced treatment'' to ``creation science'' and ``evolution science'' in the biology classroom.  It is clear from the act that religious reasons were not to be offered as support for the truth of creation science, for this would violate federal law.  Instead, the curriculum was expected to concentrate onlyu on the ``scientific evidence'' for creation science.  But was there any?  And, how precisely was creation science different from creationism?
\end{quotation}
Explain the arguments used in court to thwart Act 590 the following year.
\end{enumerate}

\end{document}
