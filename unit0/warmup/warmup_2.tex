\title{Warm-Up, August 29th 2024 (INTD262)}
\author{Dr. Jordan Hanson - Whittier College Dept. of Physics and Astronomy}
\date{\today}
\documentclass[12pt]{article}
\usepackage[margin=1.5cm]{geometry}
\usepackage{hyperref}
\usepackage{graphicx}
\usepackage{amsmath}
\begin{document}
\maketitle

\section{Introduction to \textit{Science in Latin America: A History}}

\begin{enumerate}
\item Offer some reasons why the Spaniards created the \textit{virreinatos} of Nueva Espa\~{n}a and Per\'{u} in their respective locations, with Tenochtitlan and Lima as capital cities. \\ \vspace{1cm}
\item Compare and contrast the development of science in the original thirteen colonies of the United States to the development of science in the Spanish \textit{virreinatos}. \\ \vspace{1cm}
\item Based on the reading, define the following terms: (a) \textit{telluric}, (b) \textit{endogenous}, and (c) \textit{mimesis}.  \\ \vspace{0.5cm}
\item Was there a link between the introduction of capitalism and the growth of scientific activity in Latin America, or did the growth of modern science precede capitalism? \\ \vspace{0.5cm}
\end{enumerate}

\section{Chapter 1 of \textit{Science in Latin America: A History}}

\begin{enumerate}
\item Given the definition of \textit{peripheral} scientific activity in the Introduction, can you give an example of the creating and transmission of scientific results from the periphery to the center of science? \\ \vspace{1cm}
\item Give some examples of \textit{pseudo-scientific} beliefs regarding mythical places the colonials sought in the New World.
\end{enumerate}

\end{document}
