\title{Thursday Warm-Up (INTD262)}
\author{Dr. Jordan Hanson - Whittier College Dept. of Physics and Astronomy}
\date{\today}
\documentclass[12pt]{article}
\usepackage[margin=1.25cm]{geometry}
\usepackage{hyperref}
\begin{document}
\maketitle
\small

\section{Chapter 5 of \textit{Science in Latin America}}

\begin{enumerate}
\item Match the following social reforms to those who made major contributions to the effort:
\footnotesize
\begin{itemize}
\item A: Mining reforms in late 18th-century Mexico$~~~~$1: Hip\'{o}lito Un\'{a}nue 
\item B: Medical and surgical reforms in Per\'{u}$~~~~~~~~~~~~~$2: Francisco Zea and Francisco Jos\'{e} de Caldas
\item C: Botany and naturalism in Nueva Granada$~~~~~~$3: Francisco Xavier Gamboa and Joaqu\'{i}n Vel\'{a}zquez de Le\'{o}n
\end{itemize}
\small
\item Give a brief account of how \textit{scientific nationalism} was born in Latin American nations \\ \vspace{1cm}
\item Given what you know about \textit{the scientific attitude}, interpret this quote:
\begin{quote}
A significant number of scientists formed a community in each country, and institutions especially dedicated to the fostering and teaching of science were established ... Francisco Jos\'{e} de Caldas wrote thus from his native Popay\'{a}n in 1801: `I am certain that when the nineteenth century ends, we won't have to envy the home country for its Enlightenment.'''
\end{quote} \vspace{1cm}
\item (a) When did Latin American wars of independence begin, approximately? (b) Give some examples of scientists and engineers who fought and died for their countries. (c) \textbf{Reflect:} given what we have learned about concepts like the free press and the creation of scientific institutions, what would it take for you to join in a revolution? \\ \vspace{1cm}
\item (a) Reflect on another quote from chapter 5: ``Science and technology became ... the preferred vehicle for Latin American nations to instill the notion of social equality through education, to create and develop an economy that could overcome the colonial bureaucratic and centralist vices ... to provide the nation-state with the necessary means to justify its power rationally.'' \textbf{Reflect:} Compare these ideas to the situation in our own country today.  How is STEM and STEM education related, or not, to social equality?
\end{enumerate}

\end{document}
