\title{Thursday Warm-Up (INTD262)}
\author{Dr. Jordan Hanson - Whittier College Dept. of Physics and Astronomy}
\date{\today}
\documentclass[12pt]{article}
\usepackage[margin=1.25cm]{geometry}
\usepackage{hyperref}
\begin{document}
\maketitle
\small

\section{Chapter 6 of \textit{The Scientific Attitude}}

\begin{enumerate}
\item In what decade was the germ theory of disease introduced?
\begin{itemize}
\item A: 1820s
\item B: 1840s
\item C: 1860s
\item D: 1880s
\end{itemize}
\item How long after Semmelweis's solution to childbed fever was germ theory introduced? \\ \vspace{0.25cm}
\item (a) Write down the four basic elements, according to Hippocrates, and associate them correctly with the four humors of the body. (b) Draw a connection between this origin of Western medicine and what we learned about indigenous medicine in Chapter 1 of \textit{Science in Latin America.} \\ \vspace{1cm}
\item Did physicians ever have evidence that bloodletting worked? Give examples for and against the practice. \\ \vspace{0.5cm}
\item Where did the practice of autopsies begin?  In what way does performing an autopsy fit with the scientific attitude? \\ \vspace{0.5cm}
\item Louis Pasteur first worked on the question of ``spontaneous generation.'' (a) Why is this intriguing scientifically? (b) How did he determine that bacteria could be killed?  (c) In 1871 President Garfield was hit with a bullet.  How did he die? \\ \vspace{1cm}
\item Once germ theory arose, there was a long delay in producing clinical benefit.  (a) Why? (b) Discuss the tension between official medical training/licensing and lay knowledge in democratic societies. (c) After the Flexner Report was published in 1910, what was the result?
\end{enumerate}

\end{document}
